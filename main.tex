\documentclass[a4paper,twoside,12pt,papersize, dvipdfmx]{iirthesis}
\usepackage{amsmath,amssymb,amsthm}
\usepackage{graphicx}
\usepackage{subcaption}
\usepackage{url}
\usepackage{otf}
\usepackage{minitoc}
\usepackage{bm}
\usepackage{amsmath,amssymb}
\usepackage{multirow}
\usepackage{times} % times new roman
\usepackage{pdfpages}

% 単体コンパイルを可能にするための仕込み
\newcommand{\ifdraft}{false}

\newcommand{\figref}[1]{\figurename\ref{#1}}
\newcommand{\tabref}[1]{\tablename\ref{#1}}
\renewcommand{\eqref}[1]{式~(\ref{#1})}
\newcommand{\chapref}[1]{\ref{#1}章}
\newcommand{\secref}[1]{\ref{#1}節}
\newcommand{\ssecref}[1]{\ref{#1}項}
\newcommand{\appref}[1]{付録\ref{#1}}

\begin{document}
%\includepdf[pages=-]{title.pdf}

 \pagestyle{empty}		%ページ番号を消す
 \thesistype{{令和4年度 修士論文}}
 \title{\Huge{平面内センサレスin-handケージングマニピュレーションの計画}}
 \etitle{}
 \author{}
 \eauthor{}
 \miscinfo{
 \begin{tabular}{c}
 \\
 \vspace{-3mm}
 {\Large 2022年1月24日} \\[3em]
 {\Large 指導教員 : 前田 雄介 教授} \\[2em]
 {\Large 横浜国立大学 大学院 理工学府} \\
 {\Large 機械・材料・海洋系工学専攻} \\
 {\Large 機械工学教育分野} \\
 {\Large 21NA140 中西 佑太} \\
 \end{tabular}
 }
 \maketitle

%\pagestyle{empty}
%\cleardoublepage
%\begin{center}
%    \vspace{12em}
%    \Large
%{\sffamily \bfseries \huge{}}\\
%    \vspace{2em}
%    {\LARGE by}\\
%    \vspace{2em}
%    {\LARGE 21NA140 Yuta NAKANISHI}\\
%    \vspace{9em}
%    Supervisor: Professor Yusuke MAEDA\\
%    \vspace{2em}
%    A Master's Thesis\\
%    Submitted to\\
%    the Specialization of Mechanical Engineering\\
%    in the Department of Mechanical Engineering, Materials Science, and Ocean Engineering\\
%    the Graduate School of Engineering Science of\\
%    Yokohama National University\\
%    \vspace{3em}
%    January 24, 2022
%\end{center}

\cleardoublepage	%奇数ページから始まるように改ページ

\frontmatter
\pagestyle{headings}	%ヘッダにページ番号,章・節番号,見出しを表示
\setcounter{tocdepth}{2}
\dominitoc
\tableofcontents


\mainmatter
%語句の統一に関する注意書き
%〇コンフィギュレーション空間   ×コンフィグレーション空間
%〇対象物             ×物体
%〇位置・姿勢$\phi$                               ×位置・姿勢$\theta$
%〇目標位置・姿勢         ×ゴール位置・姿勢
%探索アルゴリズムに関する説明では「ゴール」を使う


% 黒魔術
\expandafter\ifx\csname ifdraft\endcsname\relax
    \documentclass[a4paper,twoside,12pt,papersize, dvipdfmx]{iirthesis}
    \usepackage{amsmath,amssymb,amsthm}
    \usepackage{graphicx}
    \usepackage{subcaption}
    \usepackage{url}
    \usepackage{otf}
    \usepackage{minitoc}
    \usepackage{bm}
    \usepackage{amsmath,amssymb}
    \begin{document}

    \newcommand{\figref}[1]{\figurename\ref{#1}}
    \newcommand{\tabref}[1]{\tablename\ref{#1}}
    \renewcommand{\eqref}[1]{式~(\ref{#1})}
    \newcommand{\chapref}[1]{\ref{#1}章}
    \newcommand{\secref}[1]{\ref{#1}節}
    \newcommand{\ssecref}[1]{\ref{#1}項}
    \newcommand{\appref}[1]{付録\ref{#1}}
\fi


\chapter{序論}\label{chap::intro}
\minitoc

\section{研究背景}\label{sec::intro::background}
近年,ロボット技術は飛躍的に進歩し,様々な産業に対してロボットの導入が進んでいる.このようなロボットに期待されていることの一つは,人間が行っている作業を代替することである.人間が行う作業は様々あり,それ故に人間の運動技能も様々ある.その一つに,「In-handマニピュレーション」と呼ばれるものがある.これは,手の中で物体の位置・姿勢を変化させる動作のことである(\figref{}).このIn-handマニピュレーションをロボットハンドで再現しようとした研究は多くある.これらの研究の多くは力情報や位置情報を必要としており,マニピュレーション中のセンシングが不可欠である.\par

ところで,対象物の拘束手法の一つにケージング\cite{rimon1999}というものがある.これは,対象物の形状情報を基に,ロボットによる幾何学的な囲い込みで対象物を拘束する手法である.この手法は,ロボットの位置制御のみで実現可能であり,センサを必要としないという利点がある.\par

この手法を用い,ロボットハンドで対象物のケージング状態を作った後,囲いの形状を変化させていくことで,対象物の位置・姿勢を制御することができる.すなわち,ケージングに基づいたIn-handマニピュレーションが可能となる.\cite{komiyama2020}では,これを「センサレスin-handケージングマニピュレーション」と定義している.この手法では,対象物を一度捉えることができれば,以降一切のセンシングを行うことなくマニピュレーションすることができる.\par

本手法はパーツフィーダへの応用を見込んでいる.位置・姿勢にばらつきのある対象物を特定の位置・姿勢に整列できるという機能を活かして,\figref{fig::partsfeeder}のようなシステムを構築することで実現できると考える.従来のパーツフィーダと比較した場合,扱う部品形状が幅広いという利点があり,多品種少量生産傾向にある昨今に有用なパーツフィーダになると考える.\par 



\section{従来研究}\label{sec::intro::relatedresearch}
\subsection{In-handマニピュレーションに関する研究}
\subsubsection{Planning In-hand Object Manipulation with Multifingered Hands Considering Task Constraints \cite{Hertkorn2013}}
この研究では,人間の指のような多自由度のハンドを用いた,3次元の物体のin-handマニピュレーション問題を扱っている.物体は,ハンドと物体との接触(Force closure)によって把持されている.対象物とロボットハンドの接触点に注目し,この接触点を適切に変化させることで,対象物を所望の位置・姿勢へ移動させることができるとされている.そのうえで,物体とハンドの接触状態をどのように変化させていけば良いかを導出するアルゴリズムを提案していた(\figref{}).

\subsubsection{Energy Gradient-Based Graphs for Planning Within-Hand Caging Manipulation \cite{Bircher2019}}
この研究では,ハンドと対象物からなる系のエネルギー情報を定義し,このエネルギーに基づいてマニピュレーション動作を決定している.
実験には,\figref{fig::system}のような劣駆動ハンドを用いており,アクチュエータへの2入力でハンドを制御している.エネルギー情報は,任意のアクチュエータ入力に対して,離散化空間内の全ての点のエネルギーを定義式より各々算出し,エネルギーマップを生成するという形で表現している.このエネルギーマップを様々なアクチュエータ入力に対して生成しておき,エネルギー勾配が対象物の目標座標へ向いているアクチュエータ入力を随時選び,ゴールを目指すといったアルゴリズムになっている.結果は,\figref{fig::manires}のようにマニピュレーションが実現されていた.

\subsubsection{Plannar In-Hand Manipulation Via Motion Cones \cite{Chavan-dafle}}
この研究では,\figref{fig::maniT}のようにグリッパで対象物を把持し,壁等の外部環境に押し付けたタイミングで把持位置をスライドさせることで持ち替え動作を行っている.ただし,動作計画ではグリッパが固定されており,壁等の環境がプッシャとして対象物を押し出すといった見方をしている.動作計画にあたって,Motion cone (動作円錐)というものを考えている.これは,摩擦の作用を考慮したうえで対象物が1プッシュで到達可能な領域を意味する.動作計画アルゴリズムにはT-RRT*を用いており,ランダムサンプリングしたノードが親ノードのMotion cone内でかつ,効率的な動きである場合,木構造に追加する.Motion cone外であった場合は,Motion cone内に投影した点を用いて同様の判定を行っていた.この手順で枝を伸ばしていき,目標状態となったとき計画を終了している.以上の動作計画のイメージを\figref{fig::rrtimage}に示している.上記の手法に基づき,数多くの実験検証を行い,理論が正確であることが確認されていた.

\subsubsection{まとめ}
本研究と同じくIn-handマニピュレーションに取り組んだものを紹介した.これらの研究は,マニピュレーション中に対象物に関する力情報や位置情報を必要としており,センシングが不可欠である.この点が本研究との相違点となっている.


\subsection{センサレスなマニピュレーションを行っている研究}
\subsubsection{Sensorless In-Hand Manipulation by An Underactuated Robot Hand \cite{Ospina}}	\label{sec::ospina}
この研究では,センサレスかつ対象物体の幾何情報なしのIn-handマニピュレーションを提案している.\figref{fig::ohand}のようなハンドを用いており,これは2つのリンクに対して1つのアクチュエータで駆動する劣駆動ハンドとなっている.このハンド先端の円形部分で物体を挟み込み,物体操りを行っている.センサレスなマニピュレーションを実現するため,全ての把持の状態を領域化した,In-hand Manipulability Region (IHMR)というものを定義している.IHMRの範囲内でハンドを動かす限り,把持が成立し続ける.IHMRは物体形状に依存するため,様々な既知物体に対してIHMRを算出し,それらの共通領域を任意物体のIHMRとすることで,幾何情報を必要としないセンサレスin-handマニピュレーションを実現していた.

\subsubsection{動的安定把持に基づくマニピュレーション \cite{Tahara2013}\\多指ロボットハンドによる物体把持のダイナミクスと受動性 \cite{Tahara2020}}
この研究では「受動性」の概念を導入することで指が物体と接触している間についてのダイナミクスの特性を解析している.受動性とは簡単に言うと制御における安定性の指標であり,受動性を満たすとき可制御であるとは限らないが,少なくとも内部エネルギーが意図せず増加しない安定状態となっている.\figref{fig::model}のように指先を半球状のモデルで表現し,指先と物体間の接触点位置拘束と転がり速度拘束条件を考慮した運動方程式を立式していた.これに対し,系が受動性を満たすような制御入力を設計することで,安定した把持を実現していた.制御入力はロボットハンドの内界センサ値のみで算出される値であり,センサレスな把持が可能となっていた.

\subsubsection{Sensorless Pose Determination Using  Randomized Action Sequences ~\cite{Mannam}}
本研究では,\figref{fig::trayrobot}のようにロボットの手先につけたトレーをランダムに何度も傾けることで,物体姿勢のばらつきを定量化したエントロピーを減少させることができることを提案・検証している.ランダムな動作の利点として,物体形状を考える必要がなく,システムが単純であることが挙げられている.実機での検証では,\figref{fig::entropy}のように全体的にみるとエントロピーが減少しており,センサレスで対象物の姿勢をある程度同定できることを示した.

\subsubsection{まとめ}
これらの研究は,センサレスでマニピュレーションを行うという点で本研究と類似している.しかし,以下の観点で本研究の独自性・優位性が主張できる.
\cite{Ospina}は義手への応用を見込んでおり,把持を維持することに重きを置いているため,把持物体を目標位置・姿勢へ精確に運ぶといったタスクには向かない.\cite{Tahara2013}\cite{Tahara2020}も同様に把持を安定させることに注力しており,物体の目標位置・姿勢への収束は別の議論が必要である.\cite{Mannam}はセンサレスかつ物体形状に依存しない点から本研究より汎用性が高いと言える.しかし,整列に非常に長い時間がかかる.本研究のようにパーツフィーダへの応用を前提としている場合,整列時間は生産効率を左右する重要な要素であり,整列時間に優位性があるのは大きなメリットである.


\subsection{パーツフィーダ製品・研究}
%従来のパーツフィーダシステムあったほうがいいかも

\subsubsection{Parts Feeding on a Conveyor with a One Joint Robot \cite{Akella}}
この研究では,\figref{fig::flexfeeder}のように,一つの関節のみを持つロボットとベルトコンベアのみを用いてセンサレスで物体操作を行っている.
\figref{fig::sensmov}の3つの動作のみを用いており,catch動作で対象物の辺をハンドに接触させるところから始まり,物体を目標姿勢へ遷移させることができるような動作組み合わせを計算している.動作例は\figref{fig::sensres}のようになっており,異なる初期状態の対象物に対して同じ目標姿勢へと遷移させていた.
%三角形物体の成功率は20回中20回であったが,八角形物体の成功率は20回中16回となっており,原因としてtilt動作中に回転してしまったり,止まってしまったりしたことを挙げていた.
%限界は,凸多角形以外の場合は難しいことが多い,姿勢のみしか整列できない,摩擦なしを仮定,

\subsubsection{Development of Visionless Flexible Part Feeder for Handling Shock Absorbers \cite{Udhayakumar}}
この研究では,パーツフィーダの設計・製作から行う安価なシステムを提案している.shock absorberの部品に対象を絞り,次の3手順(\figref{fig::three}),(1) Singularizing Unit,(2) Orienting Unit, (3) Toppling Unit,を順に行い整列していた.(1)は物体を水平方向へ整列させるタスク,(2)は近接センサで上下判定を行い,下向きの場合に上向きに反転させるタスク,(3)は近接センサで左右の向きを検知し,目標姿勢ではない場合は部品を反転させるタスクである.このように,各手順で物体の取りうる姿勢を減らしていくことで,最終的に一つの目標姿勢へと遷移させていた

\subsubsection{まとめ}

\cite{Akella},\cite{Udhayakumar}は本研究と同様,物体操作を用いてパーツフィーダを開発している研究である.前者の研究は姿勢のみしか整列できないことが,後者は部品の汎用性がないことがそれぞれ課題としてある.本研究にはこれらを解決できる点で優位性がある.

\subsection{本研究室の従来研究}



\section{研究目的}\label{sec::intro::objective}
\cite{komiyama2020}では,対象物の位置だけでなく,姿勢(傾き)まで考慮したセンサレスin-handケージングマニピュレーションを実現した.また,それ以前の研究では扱える対象物が円形物体のみであったところを,三角形,十字型,L字型など様々な形状に対応可能となった.しかし,センサレスin-handケージングマニピュレーションの実用化を目指す上で,以下の課題が存在する.一つ目に対象物がハンドに詰まり,正常にハンドを動かせなくなる「ジャミング」が起きる点,二つ目にハンド動作計画の計算時間が長くかかる点,そして三つ目に対象物の位置決め精度が十分ではないという点である.\par

一つ目に関して,ジャミングは\cite{asamura2013}から存在するセンサレスin-handケージングマニピュレーションの課題である.\cite{komiyama2020}では,ヒューリスティクス的なアプローチで解決を試みた.これにより,ある程度はジャミングを回避できるようになったが,全ての場合に対応できるわけではなく,依然としてジャミングは起こっていた.\cite{kamikukita2021}ではベルトコンベアを用いており,ジャミング時にベルトコンベアを回すことで対象物の状態を変化させ,解消する試みがなされた.この操作を時には複数回繰り返すことで,多くの場合でジャミングを回避できるようになったが,繰り返し操作分,整列時間が長くかかる.本手法はパーツフィーダへの応用を見込んでいるので,整列時間の増加は生産効率の低下につながり,望ましくない.
二つ目に関して,一つの動作計画を生成するのに短くとも数時間,長いと数日という計算時間を要していた.これは,パーツフィーダの再立ち上げ時間に相当する.
三つ目に関して,計算時間の観点から対象物の位置決め精度が緩めに設定されていた.\par

そこで,本論文ではこれらの改善に取り組むべく,以下の3点を目標として掲げる.
\begin{itemize}
\item ジャミングの完全回避
\item ハンド動作計画の計算時間短縮
\item 位置決め精度の向上
\end{itemize}
これにより,センサレスin-handケージングマニピュレーションの有用性が高まり,整列時間,再立ち上げ時間が短く,位置決め精度も高い,パーツフィーダの開発に貢献できると考えている.

\section{本論文の構成}\label{sec::intro::configuration}
本論文の構成を以下に示す.\par
本章では,研究の背景,従来研究を述べ,研究の目的について述べた.\par
2章では,本研究室で提案されてきた平面内センサレスin-handケージングマニピュレーションについて述べる.\par
3章では,従来の動作計画の改良点と新たな動作計画手法について述べる.\par
4章では,提案手法を用いた多角形物体の平面内センサレスin-handケージングマニピュレーションの計画および実機による検証について述べる.\par
5章では,本研究の結論と今後の展望について述べる.


% 白魔術
\expandafter\ifx\csname ifdraft\endcsname\relax
    \end{document}
\fi

% 黒魔術
\expandafter\ifx\csname ifdraft\endcsname\relax
    \documentclass[a4paper,twoside,12pt,papersize, dvipdfmx]{iirthesis}
    \usepackage{amsmath,amssymb,amsthm}
    \usepackage{graphicx}
    \usepackage{subcaption}
    \usepackage{url}
    \usepackage{otf}
    \usepackage{minitoc}
    \usepackage{bm}
    \usepackage{amsmath,amssymb}
    \begin{document}

    \newcommand{\figref}[1]{\figurename\ref{#1}}
    \newcommand{\tabref}[1]{\tablename\ref{#1}}
    \renewcommand{\eqref}[1]{式~(\ref{#1})}
    \newcommand{\chapref}[1]{\ref{#1}章}
    \newcommand{\secref}[1]{\ref{#1}節}
    \newcommand{\ssecref}[1]{\ref{#1}項}
    \newcommand{\appref}[1]{付録\ref{#1}}
\fi


\chapter{センサレスin-handケージングマニピュレーション}\label{chap::sicm}
\minitoc

\section{概要}

\section{システムの全体像}\label{sec::sicm::overall}
\subsection{装置の説明}\label{subsec::sicm::equipment}
\subsection{対向型ハンド}\label{subsec::sicm::oppositehand}

\section{ハンドの動作計画}\label{sec::scim::planning}
\subsection{コンフィギュレーション空間}\label{subsec::sicm::cspace}
\subsection{RRTによるハンド動作の探索}\label{subsec::sicm::rrt}
\subsection{ケージング条件とケージングマニピュレーション可能条件}\label{subsec::sicm::caging}



% 白魔術
\expandafter\ifx\csname ifdraft\endcsname\relax
    \end{document}
\fi

% 黒魔術
\expandafter\ifx\csname ifdraft\endcsname\relax
    \documentclass[a4paper,twoside,12pt,papersize, dvipdfmx]{iirthesis}
    \usepackage{amsmath,amssymb,amsthm}
    \usepackage{graphicx}
    \usepackage{subcaption}
    \usepackage{url}
    \usepackage{otf}
    \usepackage{minitoc}
    \usepackage{bm}
    \usepackage{amsmath,amssymb}
    \begin{document}

    \newcommand{\figref}[1]{\figurename\ref{#1}}
    \newcommand{\tabref}[1]{\tablename\ref{#1}}
    \renewcommand{\eqref}[1]{式~(\ref{#1})}
    \newcommand{\chapref}[1]{\ref{#1}章}
    \newcommand{\secref}[1]{\ref{#1}節}
    \newcommand{\ssecref}[1]{\ref{#1}項}
    \newcommand{\appref}[1]{付録\ref{#1}}
\fi


\chapter{ハンド動作計画の性能向上と新アルゴリズム}\label{chap::planner}
\minitoc

\section{はじめに}\label{sec::planner::intro}
%ハンドの寸法とかPCの性能の話,動作計画で使ってる情報の説明

\section{順探索アルゴリズム}\label{sec::planner::straight}
\subsection{ゴール条件の緩和}
\subsection{$C_{\mathrm{free\_obj}}$の効率的な探索}
\subsection{位置決め精度向上アルゴリズム}


\section{逆探索アルゴリズム}\label{sec::planner::reverse}
\subsection{RRTを用いた逆探索アルゴリズム}
\secref{sec::planner::straight}の順探索では,任意の初期姿勢から対象物の目標位置・姿勢座標$P_G$を目指している.しかし,RRTはランダムサンプリングであるため$P_G$から遠ざかるような方向へも探索が進むという点で効率が悪い.そこで,対象物が目標位置・姿勢で位置決めされた時のハンド姿勢$\bm{\theta_G}$を探索の開始点とし,マニピュレーションの初期状態へと遡る方向へ探索する,逆探索アルゴリズムを提案する.

\subsubsection{探索終了条件}
逆探索であるため,探索の終了状態はマニピュレーションの初期状態に相当する.マニピュレーションの初期状態に求められることとしては,なるべく多様な対象物位置・姿勢を取り扱えるということが挙げられる.そこで,探索の終了条件を,物体が取りうる姿勢群を表す$C_{\mathrm{free\_obj}}$が閾値以上になった時と定めた.\par
このように,逆探索アルゴリズムでは探索の終了条件に座標の指定がなく,ただ$C_{\mathrm{free\_obj}}$の領域を広げればよい.そのため,順探索のような無駄な探索方向が少なくっており,探索効率の改善が見込める.

\subsubsection{ケージングマニピュレーション可能条件}
条件自体は,\secref{sec::sicm::caging}と変わらず,時間$t-1$における物体の存在領域$C'_{\mathrm{free\_obj}}$とオーバーラップしている時間$t$の$C_{\mathrm{free\_ICS}}$がただ一つ存在する場合に成立と判定される.

\subsection{位置決め最適姿勢生成アルゴリズム}
\subsection{計算時間の比較}


\section{両側探索アルゴリズム}
\subsection{RRT-Connectを用いた両側探索アルゴリズム}
\subsection{計算時間の比較}

\section{経路平滑化アルゴリズム}



% 白魔術
\expandafter\ifx\csname ifdraft\endcsname\relax
    \end{document}
\fi

% 黒魔術
\expandafter\ifx\csname ifdraft\endcsname\relax
    \documentclass[a4paper,twoside,12pt,papersize, dvipdfmx]{iirthesis}
    \usepackage{amsmath,amssymb,amsthm}
    \usepackage{graphicx}
    \usepackage{subcaption}
    \usepackage{url}
    \usepackage{otf}
    \usepackage{minitoc}
    \usepackage{bm}
    \usepackage{amsmath,amssymb}
    \begin{document}

    \newcommand{\figref}[1]{\figurename\ref{#1}}
    \newcommand{\tabref}[1]{\tablename\ref{#1}}
    \renewcommand{\eqref}[1]{式~(\ref{#1})}
    \newcommand{\chapref}[1]{\ref{#1}章}
    \newcommand{\secref}[1]{\ref{#1}節}
    \newcommand{\ssecref}[1]{\ref{#1}項}
    \newcommand{\appref}[1]{付録\ref{#1}}
\fi


\chapter{多角形のセンサレスin-handケージングマニピュレーションの計画・実行}
\minitoc

\section{実験装置}
%ベルトコンベアとか3Dプリンタで作ったとかサーボモータとか

\section{長方形のマニピュレーション}
\section{三角形のマニピュレーション}
\section{L字形のマニピュレーション}
\section{T字形のマニピュレーション}
\section{考察}




% 白魔術
\expandafter\ifx\csname ifdraft\endcsname\relax
    \end{document}
\fi

% 黒魔術
\expandafter\ifx\csname ifdraft\endcsname\relax
    \documentclass[a4paper,twoside,12pt,papersize, dvipdfmx]{iirthesis}
    \usepackage{amsmath,amssymb,amsthm}
    \usepackage{graphicx}
    \usepackage{subcaption}
    \usepackage{url}
    \usepackage{otf}
    \usepackage{minitoc}
    \usepackage{bm}
    \usepackage{amsmath,amssymb}
    \usepackage{times} % times new roman
    \begin{document}
    % ベンリダナー
    \newcommand{\figref}[1]{\figurename\ref{#1}}
    \newcommand{\tabref}[1]{\tablename\ref{#1}}
    \renewcommand{\eqref}[1]{式~(\ref{#1})}
    \newcommand{\chapref}[1]{\ref{#1}章}
    \newcommand{\secref}[1]{\ref{#1}節}
    \newcommand{\ssecref}[1]{\ref{#1}項}
    \newcommand{\appref}[1]{付録\ref{#1}}
\fi

\minitoc

\chapter{結論}\label{chap::conclusion}
\section{結論}\label{sec::conclusion::conclusion}
本研究では,センサレスin-handケージングマニピュレーションの有用性を高め,パーツフィーダを実用化するために,新たなハンド構成,対向型ハンドの提案,動作計画の高性能化の2つについて取り組んだ.前者に関して,従来のハンド構成,並列型ハンドではマニピュレーションの際,ジャミングという問題が発生していた.これに対して従来研究において様々な解決法が試みられたが,依然として一定の頻度でジャミングは発生していた.また,並列型ハンドの手元付近は構造上マニピュレーションできないという問題があった.そこで,対向型ハンドの提案により,ジャミングを限りなく減らすことができ,またどこに対象物があってもマニピュレーションできるようになった.後者に関しては,まず従来法である順探索アルゴリズムの改善を行った.パーツフィーダへ応用を考慮した探索終了条件の緩和,$\mathcal{C}_{\mathrm{free\_obj}}$の効率的な抽出により計算時間を大幅に削減した.また,動作計画より得られた最終姿勢からハンドを対象物に向けて狭めていくという手法により,位置決め精度も向上させた.次に,逆探索アルゴリズムを提案した.マニピュレーション終了時の位置決めされたハンド姿勢を探索の初期値としてユーザーが与える(,または最適姿勢生成アルゴリズムによって生成する)ことができるため,位置決め精度の高い経路が生成されることが保証されている.最後に両探索アルゴリズムを提案した.順探索アルゴリズムと逆探索アルゴリズムを組み合わせたアルゴリズムであり,マニピュレーション開始姿勢から終了姿勢へ,またその逆への非常に強いバイアスがかかるため効率の良い探索アルゴリズムとなっている.実際に計算時間の短縮が確認できた.また,順探索アルゴリズム,逆探索アルゴリズムの両方の位置決め精度向上の仕組みも引き継いでいるため,位置決め精度も同様に高くなっている.これら3つのアルゴリズムの中では,両探索アルゴリズムが計算時間,位置決め精度の面で最も優れているが,ユーザーのニーズによっては他の探索アルゴリズムが適している場合もあり,各々使い分けられるというのも本手法の有用性を高めている.\par

本動作計画で生成した経路を用いて,長方形,三角形,L字型,T字型物体に対して実機でマニピュレーションを行った.ロボットハンドには,3リンク3関節のハンドを向かい合うように2つ組み合わせた対向型ハンドを用いた.結果,いずれの対象物においても,ジャミングなく目標位置・姿勢までマニピュレーションされることを確認した.しかし,実機の誤差により計算程の位置決め精度は得られなかった.

以上,対向型ハンドの提案,動作計画の高性能化によりセンサレスin-handケージングマニピュレーションの有用性を高めることができた.これにより,ジャミングが起こらず安定し,かつ再立ち上げ時間が短く,位置決め精度が高い汎用パーツフィーダの実用化に貢献できたと考えている.

\section{今後の展望}\label{sec::conclusion::future}



% 白魔術
\expandafter\ifx\csname ifdraft\endcsname\relax
    \end{document}
\fi


\backmatter

%黒魔術
\expandafter\ifx\csname ifdraft\endcsname\relax
 \documentclass[a4paper,twoside,12pt,papersize, dvipdfmx]{iirthesis}
 \usepackage{amsmath,amssymb,amsthm}
 \usepackage{graphicx}
 \usepackage{subfig}
 \usepackage{url}
 \usepackage{otf}
 \usepackage{minitoc}
 \usepackage{bm}
 \usepackage{amsmath,amssymb}
 \usepackage{times} % times new roman
 \begin{document}
\fi

\chapter{謝辞}\label{chapter:acknowledgements}


%白魔術
\expandafter\ifx\csname ifdraft\endcsname\relax
  \end{document}
\fi

% 黒魔術
\expandafter\ifx\csname ifdraft\endcsname\relax
    \documentclass[a4paper,twoside,12pt,papersize, dvipdfmx]{iirthesis}
    \usepackage{amsmath,amssymb,amsthm}
    \usepackage{graphicx}
    \usepackage{subcaption}
    \usepackage{url}
    \usepackage{otf}
    \usepackage{minitoc}
    \usepackage{bm}
    \usepackage{amsmath,amssymb}
    \usepackage{times} % times new roman
    \begin{document}
    % ベンリダナー
    \newcommand{\figref}[1]{\figurename\ref{#1}}
    \newcommand{\tabref}[1]{\tablename\ref{#1}}
    \renewcommand{\eqref}[1]{式~(\ref{#1})}
    \newcommand{\chapref}[1]{\ref{#1}章}
    \newcommand{\secref}[1]{\ref{#1}節}
    \newcommand{\ssecref}[1]{\ref{#1}項}
    \newcommand{\appref}[1]{付録\ref{#1}}
\fi

\chapter{参考文献}\label{chap:bibliography}

\begin{thebibliography}{}
%1-intro
\bibitem[Rimon+ 1999]{rimon1999} 
E. Rimon and A. Blake:
``Caging Planar Bodies by One-Parameter Two-Fingered Gripping Systems,''
{\it The International Journal of Robotics Research}, vol.~18, no.~3, pp.~299--318, 1999.

\bibitem[込山 2021]{komiyama2021}
込山 隼:
``センサレスin-hand ケージングマニピュレーションによる物体の位置・姿勢制御'',
横浜国立大学大学院理工学府, ポートフォリオ, 2021.
  
\bibitem[Hertkorn+ 2013]{hertkorn2013}


  \bibitem[Bircher+ 2019]{bircher2019}
      %Walter G. Bircher, Andrew S. Morgan, Kaiyu Hang, Aaron M. Dollar,
	W. G. Bircher, A. S. Morgan, K. Hang, A. M. Dollar,
  	``Energy Gradient-Based Graphs for Planning Within-Hand Caging Manipulation'',
  	2019.
    
      \bibitem[Chavan-Dafle+ 2020]{chavan-dafle2020}
 % 	Nikhil Chavan-Dafle, Rachel Holladay, Alberto Rodriguez,
 	N. Chavan-Dafle, R. Holladay, A. Rodriguez,
  	``Planar In-Hand Manipulation Via Motion Cones'',
  	 The International Journal of Robotics Research,
  	 Vol.~39, No.~2--3, pp.~163--182,
  	2020.
  	
    \bibitem[Ospina+ 2020]{ospina2020}
%  	Diego Ospina, Alejandro Ramirez-Serrano,
	D. Ospina, A. Ramirez-Serrano,
  	``Sensorless In-Hand Manipulation by An Underactuated Robot Hand'',
  	 ASME. J. Mechanisms Robotics. October 2020,
  	 Vol.~12, No.~5: 051009, 
  	2020.

    \bibitem[田原 2013]{tahara2013}
	田原 健二,
  	``動的安定把持に基づくマニピュレーション'',
  	 日本ロボット学会誌,
  	 Vol.~31, No.~4, pp.~364--369,
  	2013.  	

    \bibitem[田原 2020]{tahara2020}
    	田原 健二,
  	``多指ロボットハンドによる物体把持のダイナミクスと受動性'',
  	 システム/情報/制御,
  	 Vol.~64, No.~12,pp.~477--482,
  	2020. 
  	
 	\bibitem[Mannam+ 2019]{mannam2019}
  %	Pragna Mannam, Alexander Volkov, Jr., Robert Paolini, Gregory Chirikijan, Matthew T. Mason,
  	P. Mannam, A. Volkov, Jr., R. Paolini, G. Chirikijan, M. T. Mason,
  	``Sensorless Pose Determination Using Randomized Action Sequences'',
  	 Entropy, Vol.~21, No.~2, p.~154, 
  	2019.
  	
  	\bibitem[Akella+ 2000]{akella2000}
  	S. Akella, W. H. Huang, K. M. Lynch, M. T. Mason,
  	``Parts Feeding on a Conveyor with a One Joint Robot'',
  	 Algorithmica 26,
  	 pp.~313--344,
  	2000.
  	
  	\bibitem[Udhayakumar+ 2021]{udhayakumar2021}
  	S. Udhayakumar, A. Mohan, J. Gowthamachandran,  R. Prakash, P. Shanmugam,
  	``Development of Visionless Flexible Part Feeder for Handling Shock Absorbers'',
  	 Materials, Design, and Manufacturing for Sustainable Environment, Springer Singapole, 
  	 pp.~141--154,
  	2021.  	
  	
\bibitem[上久木田 2022]{kamikukita2022}
上久木田 治毅:
``センサレス in-hand ケージングマニピュレーションに基づく汎用パーツフィーダの開発'', 
横浜国立大学理工学部, 卒業論文, 2022.
  	
\bibitem[浅村 2013]{asamura2013} 
浅村 知洋:
``ロボットハンドのためのIn-hand ケージングマニピュレーション'', 
横浜国立大学大学院工学府, 修士論文, 2013.
\end{thebibliography}

% 白魔術
\expandafter\ifx\csname ifdraft\endcsname\relax
    \end{document}
\fi


%\appendix
%% 黒魔術
\expandafter\ifx\csname ifdraft\endcsname\relax
    \documentclass[a4paper,twoside,12pt,papersize, dvipdfmx]{iirthesis}
    \usepackage{amsmath,amssymb,amsthm}
    \usepackage{graphicx}
    \usepackage{subcaption}
    \usepackage{url}
    \usepackage{otf}
    \usepackage{minitoc}
    \usepackage{bm}
    \usepackage{amsmath,amssymb}
    \begin{document}
    % ベンリダナー
    \newcommand{\figref}[1]{\figurename\ref{#1}}
    \newcommand{\tabref}[1]{\tablename\ref{#1}}
    \renewcommand{\eqref}[1]{式~(\ref{#1})}
    \newcommand{\chapref}[1]{\ref{#1}章}
    \newcommand{\secref}[1]{\ref{#1}節}
    \newcommand{\ssecref}[1]{\ref{#1}項}
    \newcommand{\appref}[1]{付録\ref{#1}}
\fi

\chapter{付録}\label{chap:appendix}


% 白魔術
\expandafter\ifx\csname ifdraft\endcsname\relax
    \end{document}
\fi


\end{document}
