% 黒魔術
\expandafter\ifx\csname ifdraft\endcsname\relax
    \documentclass[a4paper,twoside,12pt,papersize, dvipdfmx]{iirthesis}
    \usepackage{amsmath,amssymb,amsthm}
    \usepackage{graphicx}
    \usepackage{subcaption}
    \usepackage{url}
    \usepackage{otf}
    \usepackage{minitoc}
    \usepackage{bm}
    \usepackage{amsmath,amssymb}
    \begin{document}

    \newcommand{\figref}[1]{\figurename\ref{#1}}
    \newcommand{\tabref}[1]{\tablename\ref{#1}}
    \renewcommand{\eqref}[1]{式~(\ref{#1})}
    \newcommand{\chapref}[1]{\ref{#1}章}
    \newcommand{\secref}[1]{\ref{#1}節}
    \newcommand{\ssecref}[1]{\ref{#1}項}
    \newcommand{\appref}[1]{付録\ref{#1}}
\fi


\chapter{センサレスin-handケージングマニピュレーション}\label{chap::sicm}
\minitoc

\section{概要}

\section{システムの全体像}\label{sec::sicm::overall}
\subsection{装置の説明}\label{subsec::sicm::equipment}
\subsection{対向型ハンド}\label{subsec::sicm::oppositehand}

\section{ハンドの動作計画}\label{sec::scim::planning}
\subsection{コンフィギュレーション空間}\label{subsec::sicm::cspace}
\subsection{RRTによるハンド動作の探索}\label{subsec::sicm::rrt}
\subsection{ケージング条件とケージングマニピュレーション可能条件}\label{subsec::sicm::caging}



% 白魔術
\expandafter\ifx\csname ifdraft\endcsname\relax
    \end{document}
\fi
