% 黒魔術
\expandafter\ifx\csname ifdraft\endcsname\relax
    \documentclass[a4paper,twoside,12pt,papersize, dvipdfmx]{iirthesis}
    \usepackage{amsmath,amssymb,amsthm}
    \usepackage{graphicx}
    \usepackage{subcaption}
    \usepackage{url}
    \usepackage{otf}
    \usepackage{minitoc}
    \usepackage{bm}
    \usepackage{amsmath,amssymb}
    \begin{document}

    \newcommand{\figref}[1]{\figurename\ref{#1}}
    \newcommand{\tabref}[1]{\tablename\ref{#1}}
    \renewcommand{\eqref}[1]{式~(\ref{#1})}
    \newcommand{\chapref}[1]{\ref{#1}章}
    \newcommand{\secref}[1]{\ref{#1}節}
    \newcommand{\ssecref}[1]{\ref{#1}項}
    \newcommand{\appref}[1]{付録\ref{#1}}
\fi


\chapter{平面内センサレスin-handケージングマニピュレーション}\label{chap::sicm}
\minitoc

\section{概要}


\section{汎用パーツフィーダ}\label{sec::sicm::partsfeeder}
センサレスin-handケージングマニピュレーションの位置・姿勢にばらつきのある対象物を特定の位置・姿勢に整列できるという機能を活かして,本手法はパーツフィーダへ応用することを見込んでいる.本研究では,ベルトコンベアと1対のロボットハンドを用いて\figref{}のように構成している.ロボットハンドは,3つのリンク,3つの関節からなる3自由度ハンドとなっている.ここで,\figref{}のように1対のロボットハンドを並列に並べたハンド構成を並列型ハンドと呼ぶこととする.
\subsection{一連の動作の流れ\cite{kamikukita2022}}\label{subsec::sicm::flow}
パーツフィーダの動作の流れについて説明する(\figref{}).まず,ハンドを開いた状態で,ベルトコンベアにより対象物を一定時間移動させ,停止する.その後,ハンドを閉じて物体を囲み,対象物のケージング状態を作った後にマニピュレーションを行う.整列が終了すると,ハンドを開いて物体を開放し,ベルトコンベアを一定時間動かして,対象物を流出させる.\par
本来のパーツフィーダでは,物体の分離と整列が必要となるが,本研究で扱うパーツフィーダは,物体の分離は行わず,ベルトコンベア上に十分な間隔で物体があることを前提として物体の整列動作を行う.
%上の\figref{}は全部同じ参照で,パーツフィーダの流れの説明した画像.上久木田卒論のFig3.2


\subsection{対向型ハンド}\label{subsec::sicm::oppositehand}
\ref{sec::intro::objective}で述べた通り,並列型ハンドでは対象物がハンド根元付近に詰まり,ハンドが正常に動けなくなるジャミング(\figref{})が発生していた.また,並列型ハンドの構造上,ハンド根元関節付近に対象物が存在する場合,マニピュレーションできないといった問題点も存在した.\par
そこで,\figref{}のようなハンド構成,対向型ハンドを提案する.
前者の問題に関して考える.%理論を考察する
後者の問題に関しては,一方のハンドの根元側に物体があったとしても,他方のハンドで掬い取るような動作でマニピュレーションできるため,解決が見込める.\par
対向型ハンドの弱点としては,横方向に大きく運べないという点が挙げられる.しかし,横方向の移動はベルトコンベアで賄えるので,大きな問題ではない.


\section{対象物のコンフィギュレーション空間}\label{sec::sicm::cspace}
\section{ケージング条件とケージングマニピュレーション可能条件}\label{sec::sicm::caging}
\section{ハンドの動作計画}\label{sec::scim::planning}
%この従来アルゴリズムのデメリットも書く.




% 白魔術
\expandafter\ifx\csname ifdraft\endcsname\relax
    \end{document}
\fi
