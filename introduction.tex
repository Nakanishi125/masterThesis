% 黒魔術
\expandafter\ifx\csname ifdraft\endcsname\relax
    \documentclass[a4paper,twoside,12pt,papersize, dvipdfmx]{iirthesis}
    \usepackage{amsmath,amssymb,amsthm}
    \usepackage{graphicx}
    \usepackage{subcaption}
    \usepackage{url}
    \usepackage{minitoc}
    \usepackage{bm}
    \usepackage{amsmath,amssymb}
    \begin{document}

    \newcommand{\figref}[1]{\figurename\ref{#1}}
    \newcommand{\tabref}[1]{\tablename\ref{#1}}
    \renewcommand{\eqref}[1]{式~(\ref{#1})}
    \newcommand{\chapref}[1]{\ref{#1}章}
    \newcommand{\secref}[1]{\ref{#1}節}
    \newcommand{\ssecref}[1]{\ref{#1}項}
    \newcommand{\appref}[1]{付録\ref{#1}}
\fi


\chapter{序論}\label{chap:intro}
\minitoc

\section{研究背景}\label{sec::intro::background}


\section{従来研究}\label{sec::intro::relatedresearch}


\section{研究目的}\label{sec::intro::objective}
\cite{komiyama2020}では,対象物の位置だけでなく,姿勢(傾き)まで考慮したセンサレスin-handケージングマニピュレーションを実現した.また,それ以前の研究では扱える対象物が円形物体のみであったところを,三角形,十字型,L字型など様々な形状に対応可能となった.しかし,センサレスin-handケージングマニピュレーションの実用化を目指す上で,以下の課題が存在する.一つ目に対象物がハンドに詰まり,正常にハンドを動かせなくなる「ジャミング」が起きる点,二つ目にハンド動作計画の計算時間が長くかかる点,そして三つ目に対象物の位置決め精度が十分ではないという点である.\par

一つ目に関して,ジャミングは\cite{asamura2013}から存在するセンサレスin-handケージングマニピュレーションの課題である.\cite{komiyama2020}では,ヒューリスティクス的なアプローチで解決を試みた.これにより,ある程度はジャミングを回避できるようになったが,全ての場合に対応できるわけではなく,依然としてジャミングは起こっていた.\cite{kamikukita2021}ではベルトコンベアを用いており,ジャミング時にベルトコンベアを回すことで対象物の状態を変化させ,解消する試みがなされた.この操作を時には複数回繰り返すことで,多くの場合でジャミングを回避できるようになったが,繰り返し操作分,整列時間が長くかかる.本手法はパーツフィーダへの応用を見込んでいるので,整列時間の増加は生産効率の低下につながり,望ましくない.
二つ目に関して,一つの動作計画を生成するのに短くとも数時間,長いと数日という計算時間を要していた.これは,パーツフィーダの再立ち上げ時間に相当する.
三つ目に関して,計算時間の観点から対象物の位置決め精度が緩めに設定されていた.\par

そこで,本論文ではこれらの改善に取り組むべく,以下の3点を目標として掲げる.
\begin{itemize}
\item ジャミングの完全回避
\item ハンド動作計画の計算時間短縮
\item 位置決め精度の向上
\end{itemize}
これにより,センサレスin-handケージングマニピュレーションの有用性が高まり,整列時間,再立ち上げ時間が短く,位置決め精度も高い,パーツフィーダの開発に貢献できると考えている.

\section{本論文の構成}\label{sec::intro::configuration}
本論文の構成を以下に示す.\par
本章では,研究の背景,従来研究を述べ,研究の目的について述べた.\par
2章では,本研究室で提案されてきた平面内センサレスin-handケージングマニピュレーションについて述べる.\par
3章では,従来の動作計画の改良点と新たな動作計画手法について述べる.\par
4章では,提案手法を用いた多角形物体の平面内センサレスin-handケージングマニピュレーションの計画および実機による検証について述べる.\par
5章では,本研究の結論と今後の展望について述べる.


% 白魔術
\expandafter\ifx\csname ifdraft\endcsname\relax
    \end{document}
\fi
