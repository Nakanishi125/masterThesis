% 黒魔術
\expandafter\ifx\csname ifdraft\endcsname\relax
    \documentclass[a4paper,twoside,12pt,papersize, dvipdfmx]{iirthesis}
    \usepackage{amsmath,amssymb,amsthm}
    \usepackage{graphicx}
    \usepackage{subcaption}
    \usepackage{url}
    \usepackage{otf}
    \usepackage{minitoc}
    \usepackage{bm}
    \usepackage{amsmath,amssymb}
    \begin{document}

    \newcommand{\figref}[1]{\figurename\ref{#1}}
    \newcommand{\tabref}[1]{\tablename\ref{#1}}
    \renewcommand{\eqref}[1]{式~(\ref{#1})}
    \newcommand{\chapref}[1]{\ref{#1}章}
    \newcommand{\secref}[1]{\ref{#1}節}
    \newcommand{\ssecref}[1]{\ref{#1}項}
    \newcommand{\appref}[1]{付録\ref{#1}}
\fi


\chapter{多角形のセンサレスin-handケージングマニピュレーションの計画・実行}
\minitoc

\section{実験装置}
%ベルトコンベアとか3Dプリンタで作ったとかサーボモータとか

\section{長方形のマニピュレーション}
\section{三角形のマニピュレーション}
\section{L字形のマニピュレーション}
\section{T字形のマニピュレーション}
\section{考察}




% 白魔術
\expandafter\ifx\csname ifdraft\endcsname\relax
    \end{document}
\fi
