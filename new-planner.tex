% 黒魔術
\expandafter\ifx\csname ifdraft\endcsname\relax
    \documentclass[a4paper,twoside,12pt,papersize, dvipdfmx]{iirthesis}
    \usepackage{amsmath,amssymb,amsthm}
    \usepackage{graphicx}
    \usepackage{subcaption}
    \usepackage{url}
    \usepackage{otf}
    \usepackage{minitoc}
    \usepackage{bm}
    \usepackage{amsmath,amssymb}
    \begin{document}

    \newcommand{\figref}[1]{\figurename\ref{#1}}
    \newcommand{\tabref}[1]{\tablename\ref{#1}}
    \renewcommand{\eqref}[1]{式~(\ref{#1})}
    \newcommand{\chapref}[1]{\ref{#1}章}
    \newcommand{\secref}[1]{\ref{#1}節}
    \newcommand{\ssecref}[1]{\ref{#1}項}
    \newcommand{\appref}[1]{付録\ref{#1}}
\fi


\chapter{ハンド動作計画の性能向上と新アルゴリズム}\label{chap::planner}
\minitoc

\section{はじめに}\label{sec::planner::intro}
%ハンドの寸法とかPCの性能の話,動作計画で使ってる情報の説明

\section{順探索アルゴリズム}\label{sec::planner::straight}
改めて\chapref{chap::scim}のハンド動作計画の流れを振り返る.まず,入力はハンドの初期姿勢と対象物の形状情報,そして対象物の目標位置・姿勢である.これらを基に,ハンド初期姿勢からRRTによりランダムにサンプリングし,その都度対象物の形状情報から$\mathcal{C}_{\mathrm{free\_obj}}$を計算する.ケージング成立条件,ケージングマニピュレーション可能条件の満足を確認しつつ探索を進め,対象物の目標位置・姿勢から決まる探索終了条件を満たせば動作計画が完了する.出力としては,ハンドの関節角度の時系列データが得られる.\par
この動作計画アルゴリズムを順探索アルゴリズムと呼ぶこととする.この順探索アルゴリズムには,計算時間が遅い,位置決め精度が悪いという2つの課題がある.以下の項は,これらの解決方法についてであり,前者の課題に対しては,\ssecref{subsec::planner::goalcond},\ssecref{subsec::planner::dfs}で,後者の課題に対しては,\ssecref{subsec::planner::formclosure}で説明する.

\subsection{探索終了条件の緩和}\label{subsec::planner::goalcond}
\secref{sec::scim::planning}の通り,先行研究では対象物の目標位置・姿勢として$P_G (x_{\mathrm {goal}}, y_{\mathrm {goal}}, \phi_{\mathrm {goal}})$のコンフィギュレーション空間の1座標を指定していた.しかし,本手法はパーツフィーダへの応用を想定しており,この際対象物を整列後,ベルトコンベアに乗せて生産ライン方向である$x$方向へ流す.そのため,$x$方向への目標設定は必須ではない.そこで,対象物の目標位置・姿勢を$P_G (x_{\mathrm {any}}, y_{\mathrm {goal}}, \phi_{\mathrm {goal}})$と$y$,$\phi$のみを指定するようにした.\par
これに従って,探索終了条件も以下のように変更した.
\begin{gather}
\max \sqrt{(w_1\Delta x)^2 + (w_2(y-y_{\mathrm{goal}}))^2 + (w_3(\phi-\phi_{\mathrm{goal}}))^2} \leq \varepsilon \\
(x, y, \phi) \in \mathcal{C}_{\mathrm{free\_obj}} , \Delta x = x_{\mathrm {max}} - x_{\mathrm {min}}
\end{gather}
これにより,$x$方向に関しては座標の調整が必要なくなったため,先行研究より早い探索終了が見込める.

\subsubsection{計算時間の比較}

\subsection{$\mathcal{C}_{\mathrm{free\_obj}}$の効率的な探索}\label{subsec::planner::dfs}

\subsubsection{計算時間の比較}


\subsection{位置決め精度向上アルゴリズム}{\label{subsec::planner::formclosure}
計算時間の観点から,収束度$\varepsilon$は広めに設定したい.そのため,動作計画で得られたハンド最終姿勢では対象物の位置・姿勢決め精度は十分ではない.そこで,以降に示す方法によって動作計画で得られたハンド最終姿勢から微調整を加え,位置決め精度の向上を試みた.
\subsubsection{アルゴリズム}
本アルゴリズムでは,動作計画により得られた最終姿勢から更に狭まる方向へ,更にケージングが強まる方向へ微調整する.そのため,この微調整中はケージング成立条件,ケージングマニピュレーション可能条件の両者が成立していると考え,本アルゴリズムではこれを前提としている.以下,具体的なアルゴリズムについて述べる.\par
まず,対象物を目標位置・姿勢$P_G (x_{\mathrm m},y_G,\theta_G)$に仮想的に配置する.ここで,目標位置・姿勢の$x$座標に関して,\secref{subsec::planner::goalcond}の方法により,動作計画毎に位置決めされる対象物の$x$位置は変わる.そこで,$x_{\mathrm m}$には各々$\mathcal{C}_{\mathrm{free\_obj}}$の任意の$x$座標値を設定することとする.次に,片ハンドずつ対象物への距離を詰めていく.具体的には,ハンドの根元側の関節から手先側の関節の順で狭めていき,各々いずれかのハンドが対象物や他方ハンドと接触する直前で止める.この操作により,対象物の動ける範囲はさらに縮小され,位置・姿勢決め精度が向上する.\par
上記の片ハンドずつ対象物への距離を詰めていくという部分に関して,本アルゴリズムでは,左ハンド,右ハンドの順で狭めるパターンと右ハンド,左ハンドの順で狭めるパターンの2パターンを試みる.そして,より位置・姿勢決め精度が向上した方を最終的なハンド姿勢として採用する.

\subsubsection{結果}
本アルゴリズムを用いて,どの程度位置決め精度が向上するかを評価する.評価には3.3節で定義した収束度$\varepsilon$を用い,小さいほどゴールへの収束度が高く,位置・姿勢決め精度が向上したと判断する.

\section{逆探索アルゴリズム}\label{sec::planner::reverse}
\subsection{RRTを用いた逆探索アルゴリズム}
\secref{sec::planner::straight}の順探索では,任意の初期姿勢から対象物の目標位置・姿勢座標$P_G$を目指している.しかし,RRTはランダムサンプリングであるため$P_G$から遠ざかるような方向へも探索が進むという点で効率が悪い.そこで,対象物が目標位置・姿勢で位置決めされた時のハンド姿勢$\bm{\theta_G}$を探索の開始点とし,マニピュレーションの初期状態へと遡る方向へ探索する,逆探索アルゴリズムを提案する.

\subsubsection{探索終了条件}
逆探索であるため,探索の終了状態はマニピュレーションの初期状態に相当する.マニピュレーションの初期状態に求められることとしては,なるべく多様な対象物位置・姿勢を取り扱えるということが挙げられる.そこで,探索の終了条件を,物体が取りうる姿勢群を表す$C_{\mathrm{free\_obj}}$が閾値以上になった時と定めた.\par
このように,逆探索アルゴリズムでは探索の終了条件に座標の指定がなく,ただ$C_{\mathrm{free\_obj}}$の領域を広げればよい.そのため,順探索のような無駄な探索方向が少なくっており,探索効率の改善が見込める.

\subsubsection{ケージングマニピュレーション可能条件}
条件自体は,\secref{sec::sicm::caging}と変わらず,時間$t-1$における物体の存在領域$C'_{\mathrm{free\_obj}}$とオーバーラップしている時間$t$の$C_{\mathrm{free\_ICS}}$がただ一つ存在する場合に成立と判定される.

\subsection{位置決め最適姿勢生成アルゴリズム}
\subsection{計算時間の比較}


\section{両側探索アルゴリズム}
\subsection{RRT-Connectを用いた両側探索アルゴリズム}
\subsection{計算時間の比較}

\section{経路平滑化アルゴリズム}



% 白魔術
\expandafter\ifx\csname ifdraft\endcsname\relax
    \end{document}
\fi
