% 黒魔術
\expandafter\ifx\csname ifdraft\endcsname\relax
    \documentclass[a4paper,twoside,12pt,papersize, dvipdfmx]{iirthesis}
    \usepackage{amsmath,amssymb,amsthm}
    \usepackage{graphicx}
    \usepackage{subcaption}
    \usepackage{url}
    \usepackage{otf}
    \usepackage{minitoc}
    \usepackage{bm}
    \usepackage{amsmath,amssymb}
    \begin{document}

    \newcommand{\figref}[1]{\figurename\ref{#1}}
    \newcommand{\tabref}[1]{\tablename\ref{#1}}
    \renewcommand{\eqref}[1]{式~(\ref{#1})}
    \newcommand{\chapref}[1]{\ref{#1}章}
    \newcommand{\secref}[1]{\ref{#1}節}
    \newcommand{\ssecref}[1]{\ref{#1}項}
    \newcommand{\appref}[1]{付録\ref{#1}}
\fi


\chapter{ハンド動作計画の性能向上と新アルゴリズム}\label{chap::planner}
\minitoc

\section{はじめに}\label{sec::planner::intro}
%ハンドの寸法とかPCの性能の話,動作計画で使ってる情報の説明

\section{順探索アルゴリズム}\label{sec::planner::straight}
\subsection{ゴール条件の緩和}
\subsection{$C_{\mathrm{free\_obj}}$の効率的な探索}
\subsection{位置決め精度向上アルゴリズム}


\section{逆探索アルゴリズム}\label{sec::planner::reverse}
\subsection{RRTを用いた逆探索アルゴリズム}
\secref{sec::planner::straight}の順探索では,任意の初期姿勢から対象物の目標位置・姿勢座標$P_G$を目指している.しかし,RRTはランダムサンプリングであるため$P_G$から遠ざかるような方向へも探索が進むという点で効率が悪い.そこで,対象物が目標位置・姿勢で位置決めされた時のハンド姿勢$\bm{\theta_G}$を探索の開始点とし,マニピュレーションの初期状態へと遡る方向へ探索する,逆探索アルゴリズムを提案する.

\subsubsection{探索終了条件}
逆探索であるため,探索の終了状態はマニピュレーションの初期状態に相当する.マニピュレーションの初期状態に求められることとしては,なるべく多様な対象物位置・姿勢を取り扱えるということが挙げられる.そこで,探索の終了条件を,物体が取りうる姿勢群を表す$C_{\mathrm{free\_obj}}$が閾値以上になった時と定めた.\par
このように,逆探索アルゴリズムでは探索の終了条件に座標の指定がなく,ただ$C_{\mathrm{free\_obj}}$の領域を広げればよい.そのため,順探索のような無駄な探索方向が少なくっており,探索効率の改善が見込める.

\subsubsection{ケージングマニピュレーション可能条件}
条件自体は,\secref{sec::sicm::caging}と変わらず,時間$t-1$における物体の存在領域$C'_{\mathrm{free\_obj}}$とオーバーラップしている時間$t$の$C_{\mathrm{free\_ICS}}$がただ一つ存在する場合に成立と判定される.

\subsection{位置決め最適姿勢生成アルゴリズム}
\subsection{計算時間の比較}


\section{両側探索アルゴリズム}
\subsection{RRT-Connectを用いた両側探索アルゴリズム}
\subsection{計算時間の比較}

\section{経路平滑化アルゴリズム}



% 白魔術
\expandafter\ifx\csname ifdraft\endcsname\relax
    \end{document}
\fi
