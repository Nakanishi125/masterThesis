% 黒魔術
\expandafter\ifx\csname ifdraft\endcsname\relax
    \documentclass[a4paper,twoside,12pt,papersize, dvipdfmx]{iirthesis}
    \usepackage{amsmath,amssymb,amsthm}
    \usepackage{graphicx}
    \usepackage{subcaption}
    \usepackage{url}
    \usepackage{otf}
    \usepackage{minitoc}
    \usepackage{bm}
    \usepackage{amsmath,amssymb}
    \begin{document}

    \newcommand{\figref}[1]{\figurename\ref{#1}}
    \newcommand{\tabref}[1]{\tablename\ref{#1}}
    \renewcommand{\eqref}[1]{式~(\ref{#1})}
    \newcommand{\chapref}[1]{\ref{#1}章}
    \newcommand{\secref}[1]{\ref{#1}節}
    \newcommand{\ssecref}[1]{\ref{#1}項}
    \newcommand{\appref}[1]{付録\ref{#1}}
\fi


\chapter{ハンド動作計画の性能向上と新アルゴリズム}\label{chap::planner}
\minitoc

\section{順探索アルゴリズム}
\subsection{$C_{\mathrm{free\_obj}}$の効率的な探索}
\subsection{ゴール条件の緩和}
\subsection{簡易Form Closure生成アルゴリズム}


\section{逆探索アルゴリズム}
\subsection{RRTを用いた逆探索アルゴリズム}
\subsection{最適位置決め姿勢生成アルゴリズム}
\subsection{$C_{\mathrm{free\_obj}}$の効率的な探索}


\section{両側探索アルゴリズム}


\section{経路平滑化アルゴリズム}



% 白魔術
\expandafter\ifx\csname ifdraft\endcsname\relax
    \end{document}
\fi
