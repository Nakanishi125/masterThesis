%黒魔術
\expandafter\ifx\csname ifdraft\endcsname\relax
 \documentclass[a4paper,twoside,12pt,papersize, dvipdfmx]{iirthesis}
 \usepackage{amsmath,amssymb,amsthm}
 \usepackage{graphicx}
 \usepackage{subfig}
 \usepackage{url}
 \usepackage{otf}
 \usepackage{minitoc}
 \usepackage{bm}
 \usepackage{amsmath,amssymb}
 \usepackage{times} % times new roman
 \begin{document}
\fi

\chapter{謝辞}\label{chapter:acknowledgements}
本論文の執筆にあたり,前田雄介教授には多大なるご指導,ご助言を賜りました.インドへのご出張の時期と被っていながらも,何度も本論文を見て頂きました.ロボット学会の論文投稿やプレゼン発表の際も,根気強く添削してくださり,非常に感謝しています.普段の研究会やグループミーティングでは,文書の書き方や研究の進め方,研究が行き詰った時の解決方法など様々なことを学ばせていただきました.また,前田先生に質問した際には,いつも私が理解するまで丁寧に教えて頂きました.学部4年生からの3年間で確実に成長できたと感じています.\par
また,研究室メンバーの皆様にもお世話になりました.同期の坂田君,須賀君,鈴木君,高橋君,田原君には研究活動にあたって,いつもサポートしていただきました.辛いときも助け合いながらなんとか乗り越えることができました.プライベートでも東北旅行に行ったり,ピザパをしたりと充実した学生生活になりました.坂田君,鈴木君とは家に帰ってからもゲームで遊び,研究活動の良い息抜きになりました.
D3の高橋さん,M1の上久木田君,榊君,生野さん,B4の遠藤君,佐藤さん,高橋君,三上君には,研究会等において様々なアドバイスを頂きました.特に上久木田君は研究テーマが同じこともあり,軽い雑談に付き合ってもらったり,研究の相談に乗ってもらったりとお世話になりました.留学生のSaman君,Thakur君,呉さんには出身国の文化などの興味深い話を多々聞かせて頂きました.\par
最後に,このような環境を与えてくれ,いつもあたたかくサポートしてくれました家族に感謝の意を表しまして,謝辞の結びとさせていただきます.


%白魔術
\expandafter\ifx\csname ifdraft\endcsname\relax
  \end{document}
\fi
