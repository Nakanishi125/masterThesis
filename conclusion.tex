% 黒魔術
\expandafter\ifx\csname ifdraft\endcsname\relax
    \documentclass[a4paper,twoside,12pt,papersize, dvipdfmx]{iirthesis}
    \usepackage{amsmath,amssymb,amsthm}
    \usepackage{graphicx}
    \usepackage{subcaption}
    \usepackage{url}
    \usepackage{otf}
    \usepackage{minitoc}
    \usepackage{bm}
    \usepackage{amsmath,amssymb}
    \usepackage{times} % times new roman
    \begin{document}
    % ベンリダナー
    \newcommand{\figref}[1]{\figurename\ref{#1}}
    \newcommand{\tabref}[1]{\tablename\ref{#1}}
    \renewcommand{\eqref}[1]{式~(\ref{#1})}
    \newcommand{\chapref}[1]{\ref{#1}章}
    \newcommand{\secref}[1]{\ref{#1}節}
    \newcommand{\ssecref}[1]{\ref{#1}項}
    \newcommand{\appref}[1]{付録\ref{#1}}
\fi

\minitoc

\chapter{結論}\label{chap:conclusion}
\section{結論}\label{sec:conclusion:conclusion}
本研究では,センサレスin-handケージングマニピュレーションの有用性を高めるために,新たなハンド構成,対向型ハンドの提案,動作計画の高性能化の2つについて取り組んだ.前者に関して,従来のハンド構成,並列型ハンドではマニピュレーションの際,ジャミングという問題が発生していた.これに対して従来研究において様々な解決法が試みられたが,依然として一定の頻度でジャミングは発生していた.また,並列型ハンドの手元付近は構造上マニピュレーションできないという問題があった.そこで,対向型ハンドの提案により,ジャミングを限りなく減らすことができ,またどこに対象物があってもマニピュレーションできるようになった.後者に関しては,まず従来法である順探索アルゴリズムの改善を行った.パーツフィーダへ応用を考慮した探索終了条件の緩和,$\mathcal{C}_{\mathrm{free\_obj}}$の効率的な抽出により計算時間を大幅に削減した.また,動作計画より得られた最終姿勢からハンドを対象物に向けて狭めていくという手法により,位置決め精度も向上させた.次に,逆探索アルゴリズムを提案した.マニピュレーション終了時の位置決めされたハンド姿勢を探索の初期値としてユーザーが与える(,または最適姿勢生成アルゴリズムによって生成する)ことができるため,位置決め精度の高い経路が生成されることが保証されている.最後に両探索アルゴリズムを提案した.順探索アルゴリズムと逆探索アルゴリズムを組み合わせたアルゴリズムであり,マニピュレーション開始姿勢から終了姿勢へ,またその逆への非常に強いバイアスがかかるため効率の良い探索アルゴリズムとなっている.実際に計算時間の短縮が確認できた.また,順探索アルゴリズム,逆探索アルゴリズムの両方の位置決め精度向上の仕組みも引き継いでいるため,位置決め精度も同様に高くなっている.これら3つのアルゴリズムの中では,両探索アルゴリズムが計算時間,位置決め精度の面で最も優れているが,ユーザーのニーズによっては他の探索アルゴリズムが適している場合もあり,各々使い分けられるというのも本手法の有用性を高めている.\par



\section{今後の展望}\label{sec:conclusion:future}



% 白魔術
\expandafter\ifx\csname ifdraft\endcsname\relax
    \end{document}
\fi
